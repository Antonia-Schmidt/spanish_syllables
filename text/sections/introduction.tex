\documentclass[../Proposal.tex]{subfiles}

\begin{document}

%Describe in general terms the phenomenon you are studying.
% What are the primary questions/issues you will address?
This work is concerned with the effects of stress on syllable intern timing relations in Spanish. 
Different studies (e.g. \cite{gibson2019temporal}, \cite{sotiropoulou2020global}) have been looking at Spanish syllable organization under different view points and with different independent variables such as voicedness, manner of articulation of consonants and vowel. The question in this work is whether Spanish syllabic organization is influenced by stress. Here, the focus will lie on temporal stability (global vs. local syllable onset timing), although \cite{sotiropoulou2020global} have shown that this metric alone can not always be sufficient to describe syllable internal reorganization between complex and simple onsets.\\
% Provide a summary of your hypotheses, your methods, and the results you have obtained. In short, part of the introduction includes a brief summary of the entire paper.
By working with Electromagnetic Articulometry data (EMA data), I examined the variability of the duration of intervals between a postvocalic anchor and landmarks in the tautosyllabic consonants before this vowel depending on stress. Since stress does increase syllable duration, but theoretically longer durations are mitigated by calculating the relative standard deviations, I did not expect noticeable differences. Contrary to that, there seem to be some differences between stressed and unstressed syllables such as that stressed syllables show greater variability in timing relations and also sometimes exhibit locally timed onset clusters.

\end{document}